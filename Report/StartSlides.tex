\documentclass[12pt]{beamer}
\usetheme{Warsaw}
\usepackage{fontspec}
\setmainfont[Mapping=tex-text]{Times New Roman}
\usepackage[CJKchecksingle,CJKnumber]{xeCJK}
\setCJKmainfont[BoldFont={SimHei},
ItalicFont={KaiTi}]{SimSun}
\renewcommand\baselinestretch{1.2}
\usepackage{graphicx}
\usepackage{listings}
\lstset{numbers=left,numbersep=4pt,
frame=lines,framerule=1pt,basicstyle=\ttfamily\scriptsize,
}
\usepackage{amssymb}
\punctstyle{plain}
\title{人体检测及行为分析的研究}
\author{毕业设计初期答辩}
\date{}
\begin{document}

\begin{frame}
\titlepage
\end{frame}

\begin{frame}
\frametitle{课题背景}
\begin{block}{应用场景}
\begin{itemize}
\item 智能汽车
\item 监控系统
\item 高级机器人
\end{itemize}
\end{block}
\begin{block}{研究现状}
\begin{itemize}
\item 理论方面:大量特征,模型,泛型架构
\item 实验方面:性能差异大,缺乏基准参考
\end{itemize}
\end{block}
\end{frame}

\begin{frame}
\frametitle{人体检测}
\begin{block}{典型架构}
\begin{itemize}
\item 基于Haar小波特征的AdaBoost级联器
\item HOG/linSVM(方向梯度直方图/线性支持向量机)
\item NN/LRF(神经网络/局部感知域特征)
\item 联合形状-纹理检测器
\end{itemize}
\end{block}
\begin{block}{性能评测}
\begin{itemize}
\item 小分辨率/(接近于)实时处理:Haar/AdaBoost最优
\item 高分辨率/低处理速度:HOG/linSVM最优
\end{itemize}
\end{block}
\end{frame}

\begin{frame}
\frametitle{行为分析}
\begin{block}{行为描述}
\begin{itemize}
\item 底层图像信息:快速鲁棒,简单行为
\item 高层人体结构:精细复杂,复杂行为
\end{itemize}
\end{block}
\begin{block}{行为理解}
\begin{itemize}
\item 模板匹配算法:计算量少,对时间间隔敏感
\item 状态空间算法:模型复杂,屏蔽时域建模问题
\end{itemize}
\end{block}
\end{frame}

\begin{frame}
\frametitle{技术路线}
\begin{block}{参考文献}
\begin{itemize}
\item Monocular Pedestrian Detection:Survey and Experiment
\item OpenCV文档
\item ...
\end{itemize}
\end{block}
\begin{block}{编程工具}
\begin{itemize}
\item OpenCV开源视觉运算库
\item Matlab
\end{itemize}
\end{block}
\begin{block}{创新点}
针对预处理模块改进算法性能
\end{block}
\end{frame}

\begin{frame}

\end{frame}
\end{document}
