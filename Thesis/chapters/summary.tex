\chapter{总结与展望}
\section{总结}
首先,本文归纳了具有代表性的系统(尤其是人体检测系统)具有代表性的算法架构的理论,并辅以
必要的数学模型的推导。然后,基于OpenCV2以及一些运算库对选择的方法进行了简单实现。在人体
检测方面,Haar/AdaBoost\cite{haar}架构能够在实时处理条件下实现较好的识别,HOG/linSVM在
识别精度方面表现突出,但是实时性不够突出。两种架构都能达到90\%以上的命中率。在行为分析
方面,借助Laptev等人公布的时空兴趣点\cite{stip}程序包计算出的HOG描述符特征,递交给SVM
进行训练,能够实现基本的行为捕捉识别。

对于实验结果的分析,能够得到以下几点重要结论:
\begin{enumerate}
    \item[$\bullet$] \textbf{参数调整}~~对同一种架构,参数的选择对最后识别效果的影响非常大,
        这些参数主要包括划窗缩放因子,划窗尺寸,目标尺寸预判(最小目标,最大目标等),而这些
        参数都比较依赖于采用的图像数据分辨率。
    \item[$\bullet$] \textbf{数据集合}~~良好的训练数据对基于机器学习的视觉任务至关重要。
    \item[$\bullet$] \textbf{SVM核函数}~~在运用SVM学习分类时,常选择的核函数是径向基核函数,
        能取得相对于线性核函数更好的性能,但是同时会带来计算消耗的上升。
    \item[$\bullet$] \textbf{误判目标}~~从识别效果中的误判可以看出,人体作为任务目标时,竖直
        的结构更容易发生误判(遮阳伞,栏杆等)。
    \item[$\bullet$] \textbf{丢失目标}~~未检测出来的目标往往表现出和背景融合,以及人体之间的
        相互遮挡。另外一些未检测到的目标表现出和训练样本中不同的站立姿势,这也是依赖于训练数据集的。
\end{enumerate}
\section{展望}
在算法架构方面,HOG特征得到了许多完善并趋于成熟,例如\cite{boostedHOG}在HOG特征提取算法的前提下,
采用一个Fisher弱学习器进行学习,然后采用AdaBoost算法构成分类架构,取得了更好的识别精度以及实时性。
STIP特征在得到提出时\cite{stip}采用了稀疏的表达式来进行描述,之后的改进提出了密集的描述方式。

在具体实现方面,我们在上一节总结了存在的主要问题等,现在来探讨一些解决方法作为未来工作的起点。
\begin{enumerate}
    \item[$\bullet$] \textbf{参数动态选择}~~对于不同的算法架构,我们尝试取得其最佳工作状态的参数,
        然后对输入的图像进行参数方面的归一化,这样做的目的是使算法能够相对独立于输入图像分辨率的影
        响。
    \item[$\bullet$] \textbf{预处理模块}~~误判和丢失目标的原因都是复杂变化的背景造成的,引入一些
        预处理模块作为系统前级,或许能够在很大程度上改善识别效果,比如背景差分,直方图匹配等。
    \item[$\bullet$] \textbf{实时处理}~~要提高实时处理性能,需要从减少计算复杂度入手。总结采用
        的方法可以发现,划窗方法以及特征计算耗时突出。可以通过引入一些先验知识(相机视角,肤色模型)
        等来进行目标预筛选,让分类器处理更有可能是目标的对象区域,这样能够提高实时性能。
\end{enumerate}
    
