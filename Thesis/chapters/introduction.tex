\chapter{引言}
\section{背景介绍}
机器视觉所研究的一个主要问题是:如何让机器视觉系统具备“计划”和“决策能力”?从而使之完成
特定的技术动作(例如:智能汽车在检测到前方有行人时进行碰撞规避)。机器视觉系统作为一个
感知器,为动作的决策提供信息。人体检测及行为分析是当前机器视觉领域的热点课题之一,对图
像进行人体检测并对检测出的人类对象进行分析在诸多应用场景(智能汽车,高级机器人,监控系统
)中是关键的一个环节。

从应用的角度来看,例如智能汽车这样的应用场景对人体检测及行为分析的性能要求非常严格(如智
能汽车直接关乎驾驶环境下的安全程度),实际应用中表现出来的性能指标稍微差一点便绝对不能投
入实际使用。并且,机器视觉系统应该体现出相对于人类来完成某一工作时更加卓越的表现,如提高
无人监控系统相对于传统手工登记的快速性,高级机器人相对于传统工种的敏锐性,智能汽车相对
于手动驾驶的安全性等。所以,人体检测及行为分析是具有广阔前景同时非常具有挑战性的一个领域。

从机器视觉的角度来说,人体检测及行为分析是一项困难的任务。首先是因为显式模型的匮乏,机
器很难以人的思维方式去对视野中的目标进行捕获和判断,机器学习技术选择从训练样本中学习隐
式模型。其次是分类架构的性能问题,人体检测及行为分析从本质上来说是多级对象分类问题的一
个案例,对特征模型进行低风险和高精确度的感知和理解进而做出分类需要鲁棒的分类架构。

\section{研究现状}
人体检测及行为分析在过去数年内吸引了相当数量的来自计算机视觉社区的研究兴趣。许多技术理
论以特征模型和泛型架构的形式被提出,使得视觉识别领域从分类玩具对象实例发展到识别自然图
像中的多种类的对象和场景,有了显著的进步,这受益于新的鲁棒的图像描述和分类方法。

另外,在数据集合方面,收集泛数据集合的工作量是相当大的,这在某种程度上造成了数据集合的
匮乏(特别是实时和自然条件下的数据集合)。各种技术理论的实现离不开数据集合,数据集合的质
量在很大程度上决定了该种技术架构能否达到理论提出的饱和性能。本节稍后部分将会对常用的数
据集合进行概述。

\subsection{主要部件}
人体检测及行为分析系统的主要组件可以分为两部分:初始对象假设生成/兴趣区(ROI)选择,分类/模型匹配
。
\subsubsection{假设生成}
获取初始对象假设生成最简单的方法是采用划窗技术,划窗技术设定检测窗口的尺寸和位置在图像上进行
位移,这样往往会造成计算消耗超出处理容限\cite{DT2005}。通过基于已知的关于目标对象类的先验信息限制搜索空间或是
将划窗技术与递增复杂度的级联分类器\cite{boostedHOG},\cite{haar}结合可以显著提高处理速度。

除了划窗技术外,其他获取初始对象假设的技术从图像数据中提取特征。例如在静态镜头的监督学习方法中
通常会采用背景差分技术。另外一种技术采用兴趣点检测器\cite{stip},这种技术受启发于通常产生于对象
边界的图像明亮度函数间断点所包含的大量信息。
\subsubsection{模型匹配}
获取初始对象假设之后,需要进行验证(分类),这需要引入采用多种空域和时域线索的行人外貌模型,包括
生成模型和判别模型\cite{classificationmethods}。生成模型和判别模型的主要区别在与后验概率估计方法。

\textbf{生成模型}~~依据类条件密度函数$p(X|Y)$对目标对象外貌进行建模,与类先验概率$p(X)$联系起来
采用贝叶斯方法可以推导出目标对象后验概率。
\begin{equation}
    p(Y|X)=\frac{p(X|Y)\times{}p(Y)}{p(X)}
\end{equation}
生成模型从统计的角度表述数据分布,反映同类数据本身的相似度,而不关心各个目标对象类之间
的判别边界。

\textbf{判别模型}~~与生成模型不同,判别模型直接近似估计贝叶斯最大后验概率进行决策,从
训练实例中不同目标对象类之间获取判别函数的参数。这种方法不考虑样本的产生模型,即不关心
训练数据本身的特性。直接研究预测模型,寻找不同对象类之间的分类面,反映的是异类数据之间
的差异。典型的判别模型包括k近邻算法,感知机,决策树,支持向量机等。
\subsection{特征描述与提取算法}
在像素强度上进行局部滤镜操作目前被广泛采用。非适应型Haar小波特征由Papagerorgiou
和Poggio用于描述图像。完备的特征词典代表了不同区域,尺寸以及方向的局部像素
差异,其简洁性和快速性使得Haar小波特征得到普及。自动化的特征选择过程即多种AdaBoost类算法
被用于选择最具区别性的特征子集\cite{haar},这实际上是一种针对分类任务的特征最优化。

类似的,其他特殊的空间特征架构被引入用于在训练过程中产生适应于底层数据的特征集合,即
局部感知域\cite{nnlrf},受启发于人类视觉皮层的神经结构。近来的研究经验性地表明在人体分类方面自适应
性的局部感知域相对于非适应性的Haar小波特征所具有的优越性。

采用其它思想的特征描述方法聚焦于之前提及的局部边沿结构模型图像明亮度函数的间断性。从局部图像区域
中计算图像梯度方向直方图并进行标准化在密集的\cite{DT2005}(HOG)和稀疏的\cite{sift}(SIFT)特征表达式中
得到普及。密集的方向梯度直方图采用固定大小的图像单元进行计算。稀疏性特征描述方法采用兴趣点检测器
进行预处理\cite{sift},\cite{stip}。

作为以上提到的空域特征的扩展,时空域特征得到提出。通过结合时域强度特征差异,Haar小波特征
和HOG特征都可以得到扩展。报告显示时空域特征相对于空域特征具有优越性,但同时需要处理时域
对齐问题。
\subsection{学习算法}
判别模型旨在从特征空间的模式类别中学习最优判决边界。从这个目标出发,在人体检测的背景下,
多层神经网络通过调整网络参数来实现最小误差判据,被应用于自适应局部感知域特征\cite{nnlrf}。
另外,支
持向量机\cite{nsl},\cite{PRML}已成为解决模式分类问题的有力工具,与神经网络相比,支持向量机最大化决策边界实现
不同的目标对象类之间的区分度最大化,已被应用于和多类特征集合进行组合\cite{DT2005},\cite{stip}。
非线性支持向量机相对于线性支持向量机,采用非线性核函数将样本映射到高维空间中,在获取性能
提升的同时带来明显的计算消耗提升。

特征自动提取的AdaBoost算法\cite{adaboost}用于通过弱分类器的线性加权组合来构建强分类器,每个弱分类器
对单一特征设置门限。Viola等人针对人体目标问题提出了改进的级联检测器\cite{haar}并得到许多
人的改进,在训练过程中,每一层都聚焦于处理前一层的错误,随着级联器复杂度递增,分类精确度
得到提升。
\subsection{数据集合}
数据集合规模是相当庞大的,通常包含数以千计的训练样本以及大量的测试图像。在人体检测及行为
分析方面,数据集合在复杂度(动态变化的背景)和安全保护应用(碰撞或是行为风险评估)的情景真实性
方面要求非常严格。

\threelinetable[htbp]{table1}{0.93\textwidth}{lccr}{现有数据集合概览}
{数据集&训练样本&测试集合&说明\\
}{MIT CBCL&924(裁剪)&无&前后视角\\
INRIA Person&2416(裁剪)/1218(整图)&1132(裁剪)/453(整图)&彩色图像\\
Daimler Pedestrain&15660(裁剪)/6744(整图)&21790(整图)&车载采集标注\\
KTH Action&600(视频)&~&6类行为\\
Hollywood Human&233(自动)/219(手工)&211(手工)&采集32部电影\\
}{
}
从表\ref{table1}中可看出,在人体检测方面,与其他数据集合相比,Daimler提供的行人数据库
\footnote{数据库免费提供给学术或是研究用途,可从\url{http://www.science.uva.nl/research/isla/
downloads/pedestrians/index.html}下载。}的大小和复杂度更能够使实验方面得出有意义的结论。
训练图像在不同的时间和地点进行记录,除了人体都是保持直立姿势外,没有明亮度,姿势或是衣着
方面的限制,图\ref{daimler}给出了一些示例。

\pic[htbp]{Daimler行人数据库}{width=0.6\textwidth}{daimler}

在行为分析方面,多数已有的人类行为识别数据集合(如\cite{kth})只提供了处于控制和简化的场景
设定下记录的较少的行为类别,这和现实生活应用要求的处理包含具有个体差异的人类行为的自然视频有
很大的差异,这些个体差异来自于表情,姿势,动作和衣着,透视效果和镜头运动,明亮度差异,
以及场景遮挡变化等。\cite{stip}处理了当前数据集的限制问题,采集现实视频中的人类行为
样本,引入自动标注电影中的人类行为的基于剧本对齐和文本分类的方法,图\ref{hollywood}给出
了一些示例。

\pic[htbp]{Hollywood行为数据库三类样本:接吻,接电话,走出汽车}{width=0.6\textwidth}{hollywood}
\subsection{存在问题}
\textbf{评估基准}~~尽管技术理论的发展非常迅速,在实验方面,情况并不是那么乐观,
不同技术架构的性能在不同的报告下差异巨大(甚至达到几个数量级)。这源自于缺少一个公认的评估基准。

\textbf{维度灾难}~~特征矢量的维度上升的同时带来了明显的计算消耗和内存限制等问题,需要在
足够描述外貌特征和实际可行性之间寻求权衡。

\textbf{实际应用}
综述\cite{survey}中报告的性能结果与实际应用要求相比,仍存在数量级层面的差异。这可能来自训
练数据集的有限大小(是否足以覆盖特征空间),测试场景的类型差异(背景,明亮度,分辨率)等。人体
检测及行为分析系统方面还有相当多的工作需要完成。
\section{本文研究内容}
\cite{survey}报告指出,在人体检测方面,基于Haar小波特征的AdaBoost级联器\cite{haar}在低分辨
率图像和(接近于)
实时处理的条件下性能最优,而HOG特征和线性支持向量机的组合\cite{DT2005}在中等分辨率和低处理
速度限制的条件下表现突出。本文将着重于研究Haar/AdaBoost和HOG/SVM架构在人体检测中
的应用。

在行为分析方面,\cite{stip}处理了数据集合的限制,并将空域兴趣点扩展到时空域表示,在现有的
数据库以及电影测试样本中表现突出,本文将着重于研究STIP/SVM架构在行为分析中的应用。

在具体实现方面,文章研究采用OpenCV2开源运算库\cite{opencv}对研究的架构进行实现,
参考了\cite{opencvdoc}提供的文档以及\cite{opencvbook}书籍。
\section{本文结构}
文章分为以下几个章节:

第一章引言部分介绍课题研究背景,研究现状和存在的难题。其中研究现状部分中对具有代表性的特征
模型和分类架构进行了概述。

第二章较为详细地介绍人体检测方面Haar/AdaBoost和HOG/SVM架构的基本原理,将分为特征描述(Haar
特征和HOG特征)以及分类架构(AdaBoost级联器和SVM)两个部分进行阐述。

第三章较为详细地对行为分析方面的时空兴趣点特征描述和SVM的组合的基本原理。同时,这一章将引入
对样本自动提取技术详细描述。

第四章是一个关于具体实现的介绍。主要包含OpenCV2中提供的特征描述符的接口参数说明以及具体实现
方法等,在章节末尾会简要呈现实现效果。

第五章总结了研究成果和实验经验,探讨了目前存在的问题和不足以及未来有可能需要完成的工作。
