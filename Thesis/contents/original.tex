\chapter{Monocular Pedestrian Detection:Survey and Experiments}
\begin{center}
    Markus Enzweiler,Dariu M. Gavrila\footnote{文献引用标号以原文为准。}
\end{center}
\begin{center}
    \textbf{Abstract}
\end{center}

Pedestrian detection is a rapidly evolving area in computer vision with key applications in intelligent vehicles, surveillance,
and advanced robotics. The objective of this paper is to provide an overview of the current state of the art from both methodological
and experimental perspectives. The first part of the paper consists of a survey. We cover the main components of a pedestrian
detection system and the underlying models. The second (and larger) part of the paper contains a corresponding experimental study.
We consider a diverse set of state-of-the-art systems: wavelet-based AdaBoost cascade , HOG/linSVM , NN/LRF , and
combined shape-texture detection . Experiments are performed on an extensive data set captured onboard a vehicle driving
through urban environment. The data set includes many thousands of training samples as well as a 27-minute test sequence involving
more than 20,000 images with annotated pedestrian locations. We consider a generic evaluation setting and one specific to pedestrian
detection onboard a vehicle. Results indicate a clear advantage of HOG/linSVM at higher image resolutions and lower processing
speeds, and a superiority of the wavelet-based AdaBoost cascade approach at lower image resolutions and (near) real-time
processing speeds. The data set (8.5 GB) is made public for benchmarking purposes.
\section{Introduction}
Finding 
people in images is a key ability for a variety of
important applications. In this paper, we are concerned
with those applications where the human body to be
detected covers a smaller portion of the image, i.e., is visible
at lower resolution. This covers outdoor settings such as
surveillance, where a camera is watching down onto a
street, or intelligent vehicles, where an onboard camera
watches the road ahead of possible collisions with pedes-
trians. It also applies to indoor settings such as a robot
detecting a human walking down the hall. Hence our use of
the term “pedestrian” in the remainder of the paper, rather
than the more general “people” or “person.” We do not
consider more detailed perception tasks such as human
pose recovery or activity recognition.

Pedestrian detection is a difficult task from a machine
vision perspective. The lack of explicit models leads to the
use of machine learning techniques, where an implicit
representation is learned from examples. As such, it is an
instantiation of the multiclass object categorization problem
(e.g., [79]). Yet the pedestrian detection task has some of its
own characteristics, which can influence the methods of
choice. Foremost, there is the wide range of possible
pedestrian appearance, due to changing articulated pose,
clothing, lighting, and background. The detection compo-
nent is typically part of a system situated in a physical
environment, which means that prior scene knowledge
(camera calibration, ground plane constraint) is often
available to improve performance. Comparatively large
efforts have been spent to collect extensive databases; this
study, for example, benefits from the availability of many
thousands of samples. On the other hand, the bar regarding
performance and processing speed lies much higher, as we
will see later.

Pedestrian detection has attracted an extensive amount of
interest from the computer vision community over the past
few years. Many techniques have been proposed in terms of
features, models, and general architectures. The picture is
increasingly blurred on the experimental side. Reported
performances differ by up to several orders of magnitude
(e.g., within the same study [74] or [39] versus [74]). This
stems from the different types of image data used (degree of
background change), the limited size of the test data sets, and
the different (often, not fully specified) evaluation criteria
such as localization tolerance, coverage area, etc.

This paper aims to increase visibility by providing a
common point of reference from both methodological and
experimental perspectives. To that effect, the first part of the
paper consists of a survey, covering the main components of a
pedestrian detection system: hypothesis generation (ROI
selection), classification (model matching), and tracking.

The second part of the paper contains a corresponding
experimental study. We evaluate a diverse set of state-of-the-
art systems with identical test criteria and data sets as follows:
\begin{enumerate}
\item[$\bullet$] Haar wavelet-based AdaBoost cascade;
\item[$\bullet$] histogram of oriented gradient(HOG) features combined with a linear SVM;
\item[$\bullet$] neural network using local receptive fields; and
\item[$\bullet$] combined hierarchical shape matching and texture-based NN/LRF classification
\end{enumerate}

In terms of evaluation, we consider both a generic and an
application-specific test scenario. The generic test scenario
is meant to evaluate the inherent potential of a pedestrian
detection method. It incorporates no prior scene knowledge
as it uses a simple 2D bounding box overlap criterion for
matching. Furthermore, it places no constraints on allow-
able processing times (apart from practical feasibility). The
application-specific test scenario focuses on the case of
pedestrian detection from a moving vehicle, where knowl-
edge about camera calibration, location of the ground plane,
and sensible sensor coverage areas provide regions of
interest. Evaluation takes place in 3D in a coordinate system
relative to the vehicle. Furthermore, we place upper bounds
on allowable processing times (250 ms versus 2.5 s per
frame). In both scenarios, we list detection performance
both at the frame and trajectory levels.

The data set is truly large-scale; it includes many tens of
thousands of training samples as well as a test sequence
consisting of 21,790 monocular images at 640 Â 480 resolu-
tion, captured from a vehicle in a 27-minute drive through
urban traffic. See Table 1. Compared to previous pedestrian
data sets, the availability of sequential images means that
also hypothesis generation and tracking components of
pedestrian systems can be evaluated, unlike with [28], [46],
[49]. Furthermore, the data set excels in complexity
(dynamically changing background) and realism for the
pedestrian protection application onboard vehicles.

The scope of this paper is significantly broader than our
previous experimental study [49], which focused on
pedestrian classification using low-resolution pedestrian
and nonpedestrian cutouts ($18\times36$ pixels). Here, we
evaluate how robust and efficient pedestrians can be
localized in image sequences in both generic and applica-
tion-specific (vehicle) settings. Among the approaches
considered, we include those that rely on coarse-to-fine
image search strategies, e.g., see Section 4.4.

The remainder of this paper is organized as follows:
Section 2 surveys the field of monocular pedestrian detection.
After introducing our benchmark data set in Section 3,
Section 4 describes the approaches selected for experimental
evaluation. The result of the generic evaluation and the
application-specific pedestrian detection from a moving
vehicle are listed in Section 5. After discussing our results
in Section 6, we conclude the paper in Section 7.
