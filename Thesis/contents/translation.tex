\chapter{单眼视觉行人检测:综述和实验}
\begin{center}
    Markus Enzweiler,Dariu M. Gavrila\footnote{文献引用标号以原文为准。}
\end{center}
\begin{center}
    \textbf{摘要}
\end{center}

行人检测是计算机视觉中快速发展的一个领域,在智能汽车,监控系统和高级机器人等方面
具有关键性应用.这篇文章的目的是同时从方法学和实验学视角提供一个关于(该领域)目   
前的技术发展水平的综述.文章的第一部分是一个概览.这一部分涵盖了行人检测系统的主要
组件和底层模型.文章的第二(同时也是占更大比重的)部分是一个相关的实验研究.我们
考察了目前具有代表性的多种系统模型:基于小波的AdaBoost级联器[74],
HOG/linSVM[11],NN/LRF[75],和联合形状-纹理检测器
[23].实验采用城市环境行驶车辆捕获的泛数据集.数据集包含了多达数以
千计的训练样本以及一个27分钟的包含了超过20,000张具有行人位置注释的
图像的测试序列.我们考察了一般评估设定和车载系统行人检测的特殊评估设定.实验结果
表明HOG/linSVM在高分辨率和低处理速度条件下的明显优势,同时,基于小波的AdaBoost级联器
在较低分辨率和(接近于)实时处理速度条件下的优势.数据集(8.5GB)公诸于众满足
基准测试的目的.
\section{引言}
对图像进行人体检测是诸多重要应用的关键环节.在这篇文章中,
我们只关注那些待检测人体只占图像较小部分的应用设定,即在低分辨
率下的可视对象.这包括了诸多户外设定,例如:摄像头俯视监视街道的监控
系统,车载摄像头监视前方道路的行人以评估潜在碰撞可能性的智能汽车.人
体检测同时也可应用于诸如机器人检测过道上行人的室内设定.因此本文
剩余部分我们都用``行人''这个词,而不是更泛义的``人''.我们不考察例如
人类姿态复原或是行为识别等更具体的任务.

行人检测从机器视觉的角度来说是一项困难的任务.由于显式模型的匮乏,
我们选择使用从实例样本中学习隐式表示的机器学习技术.就其本身而言,行人
检测是多级对象分类问题的一个案例(例如,[79]).然而行人检测任务
具有一些自己的特征,这会影响到选择的方法.首先,存在很多可能的行人
外形,依赖于姿势,穿着,光照条件以及背景等因素.检测装置通常是装
配在物理环境中的系统的一部分,这意味着先验知识(相机校正,地平面约束)
能够提升性能.收集泛数据集是相当耗费精力的;这项研究就得益于
已有的数以千计的样本.另一方面,我们将会看到,行人检测对于性能和处理
速度的门槛要相对高出许多.

行人检测在过去数年内吸引了相当数量来自计算机视觉社区的研究兴趣.
许多技术理论以特征,模型和泛型架构的形式被提出.在实验方面情况并不是
这么乐观.报告中提及的性能往往相差几个数量级(例如,[74]内部
性能差异或[39]与[74]相比的性能差异).这源于采用图像数据
类型差异(背景变化程度),测试数据集有限大小,以及不同(通常未被
详细规定)的评估标准:定位容差,覆盖范围等.

这篇文章旨在同时从方法论角度和实验学角度,通过提供一个通用的基准参考
点来提高性能评估的可视化程度.为了达到这个目的,文章的第一部分将会是一个综述,
涵盖了行人检测系统的主要组件:假设生成(ROI选择),分类(模
型匹配),以及目标跟踪.文章第二部分是一个相关的实验研究.我们用之后提及的相
同标准和数据集评估多种具有代表性的系统:
\begin{enumerate}
    \item[$\bullet$]基于Haar小波的AdaBoost级联器[74];
    \item[$\bullet$]方向梯度直方图(HOG)特征与线性支持向量机的组合[11];
    \item[$\bullet$]采用局部感知域特征的神经网络(NN/LRF) [75]; 
    \item[$\bullet$]分层形状匹配和基于纹理的NN/LRF分类器的组合[23].
\end{enumerate}

在评估方面,我们同时考察一般测试场景和限定特殊应用的测试场景.
一般测试场景即评估一种行人检测方法的固有潜力.由于一般测试场景
采用一个简单2维边界盒重叠标准用于匹配,不会引入先验知识.此外,
它对可容许处理时间没有任何限制(不考虑实际可行性).限定特殊应用的
测试场景聚焦于车载行人检测系统的应用,在这种测试场景下关于相机
校正,地平面定位以及可感知传感器覆盖范围的信息提供了感兴趣区域(ROI).
评估在涉及车辆的三维坐标系中进行.此外,我们对可容许处理时间设定了
限制(每帧250毫秒与每帧2.5秒).在两种测试场景中,我们都列出了在帧层面和
轨线层面的检测性能.

数据集事实上是相当大规模的;它包括数以万计的训练样本以及
历时27分钟的从城市交通行驶中采集的由21,790张分辨率
为$640\times480$单眼图像的测试序列.如TABLE~1所示.与之前行人
数据相比,序列图像的存在意味着假设生成和行人检测系统的跟踪组件
同样可以得到评估,而不像[28],[46],[49]那样.
此外,数据集在复杂度(动态变化背景)和在车载行人碰撞保护应用中的情景真
实性方面表现卓越.

这篇文章的视野相比我们之前聚焦于低分辨率($18\times36像素$)
行人和非行人轮廓图像的行人分类实验研究[49]有显著拓宽.
这里,我们对在一般场景和限定特殊应用(车载)场景设定下的图像序列
中定位行人的鲁棒性和有效性进行评估.在考察的方法中,我们包括了依赖
于由粗到精的图像搜索策略的方法,例如,见Section~4.4.

文章下文组织如下:第2节对单眼视觉行人检测进行综述.第3节
介绍我们的基准测试数据集,之后,第4节描述用于实验评估方法.一般评估
和限定特殊应用(车载)评估的结果将在第5节中陈列.在第6节中讨论结果后,
我们在第7节中作出结论.

